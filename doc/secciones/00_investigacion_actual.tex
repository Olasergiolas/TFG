%%%%%%%%%%%%%%%%%%%%%%%%%%%%%%%%%%%%%%%%%
% Short Sectioned Assignment LaTeX Template Version 1.0 (5/5/12)
% This template has been downloaded from: http://www.LaTeXTemplates.com
% Original author:  Frits Wenneker (http://www.howtotex.com)
% License: CC BY-NC-SA 3.0 (http://creativecommons.org/licenses/by-nc-sa/3.0/)
%%%%%%%%%%%%%%%%%%%%%%%%%%%%%%%%%%%%%%%%%

% \documentclass[paper=a4, fontsize=11pt]{scrartcl} % A4 paper and 11pt font size
\documentclass[11pt, a4paper, openany]{book}
\usepackage[T1]{fontenc} % Use 8-bit encoding that has 256 glyphs
\usepackage[utf8]{inputenc}
\usepackage{fourier} % Use the Adobe Utopia font for the document - comment this line to return to the LaTeX default
\usepackage{listings} % para insertar código con formato similar al editor
\usepackage[spanish, es-tabla]{babel} % Selecciona el español para palabras introducidas automáticamente, p.ej. "septiembre" en la fecha y especifica que se use la palabra Tabla en vez de Cuadro
\usepackage{url} % ,href} %para incluir URLs e hipervínculos dentro del texto (aunque hay que instalar href)
\usepackage{graphics,graphicx, float} %para incluir imágenes y colocarlas
\usepackage[gen]{eurosym} %para incluir el símbolo del euro
\usepackage{cite} %para incluir citas del archivo <nombre>.bib
\usepackage{enumerate}
\usepackage{hyperref}
\usepackage{graphicx}
\usepackage{tabularx}
\usepackage{booktabs}

\usepackage[table,xcdraw]{xcolor}
\hypersetup{
	colorlinks=true,	% false: boxed links; true: colored links
	linkcolor=black,	% color of internal links
	urlcolor=cyan		% color of external links
}
\renewcommand{\familydefault}{\sfdefault}
\usepackage{fancyhdr} % Custom headers and footers
\pagestyle{fancyplain} % Makes all pages in the document conform to the custom headers and footers
\fancyhead[L]{} % Empty left header
\fancyhead[C]{} % Empty center header
\fancyhead[R]{Sergio García Cabrera} % My name
\fancyfoot[L]{} % Empty left footer
\fancyfoot[C]{} % Empty center footer
\fancyfoot[R]{\thepage} % Page numbering for right footer
%\renewcommand{\headrulewidth}{0pt} % Remove header underlines
\renewcommand{\footrulewidth}{0pt} % Remove footer underlines
\setlength{\headheight}{13.6pt} % Customize the height of the header

\usepackage{titlesec, blindtext, color}
\definecolor{gray75}{gray}{0.75}
\newcommand{\hsp}{\hspace{20pt}}
\titleformat{\chapter}[hang]{\Huge\bfseries}{\thechapter\hsp\textcolor{gray75}{|}\hsp}{0pt}{\Huge\bfseries}
\setcounter{secnumdepth}{4}
\usepackage[Lenny]{fncychap}


\begin{document}

	\chapter{Estado del arte}

    En este breve documento de carácter temporal se resumirán ideas y contenidos investigados 
    hasta el momento sobre el \textbf{Estado del arte} del \textit{Fuzzing en dispositivos IoT}.

    \section{Resúmen}
    En la actualidad el Fuzzing en dispositivos IoT es un campo de investigación considerablemente
    jóven

    \section{Puntos clave}
    \begin{itemize}
        \item Actualmente, la gran mayoría de software dedicado específicamente a fuzzing IoT 
        provienen de papers y son proyectos poco documentados y abandonados.
        \item Las principales técnicas generales más utilizadas son:
        \begin{enumerate}
            \item Rehosting
            \item Fuzzing de protocolo de comunicación (MQTT y similares).
            \item Fuzzing de peticiones HTTP a portal web de administración.
            \item Fuzzing a snippets de código con Unicorn Engine.
            \item Binary-only fuzzing en QEMU.
        \end{enumerate}
        \item La mayoría de firmware IoT no es Open Source y recientemente vienen cifrados.
        \item Mayoritariamente aplicaremos Black-box fuzzing.
    \end{itemize}

    \section{Herramientas}

    \section{Estructuración}
    En esta sección se comentará de forma general qué se plantea tratar en el TFG actualmente.
    
    \begin{itemize}
        \item Introducción al fuzzing en general
        \item Estado del arte
        \item Planificación
        \item Fuzzing targets
        \item Aplicación práctica de técnicas de fuzzing IoT
        \begin{itemize}
            \item \textbf{Rehosting + Fuzzing HTTP del portal web}: Usar emulación de firmware
            para levantar el portal web del dispositivo y realizarle fuzzing con las peticiones HTTP.
            \item \textbf{Fuzzing de protocolos de comunicación IoT como MQTT o CoAP}
            \item \textbf{Fuzzing de binarios del firmware (AFL QEMU)}: Uso de binwalk para extraer
            binarios propios del firmware a los que aplicar black-box fuzzing con AFL.
            \item \textbf{Ingeniería inversa + fuzzing de snippets de código (Unicorn Engine)}
        \end{itemize}
        \item Resultados
        \item Conclusiones
    \end{itemize}

    \section{Recursos útiles}
\end{document}
