\thispagestyle{empty}

\begin{center}
{\large\bfseries Introducción al Fuzzing, uso y estrategias de Fuzzing
para encontrar vulnerabilidades en dispositivos IoT}\\
\end{center}
\begin{center}
Sergio García Cabrera\\
\end{center}

%\vspace{0.7cm}

\vspace{0.5cm}
\noindent{\textbf{Palabras clave}: \textit{Fuzzing, IoT, Emulación, Vulnerabilidad, Sistemas empotrados,
,Black-box fuzzing, AFL++, QEMU, Open Source.}
\vspace{0.7cm}

\noindent{\textbf{Resumen}\\
Dado el creciente incremento de dispositivos conectados a internet a nuestro alrededor y la cada vez más 
sensible información que estos manejan, es una necesidad innegable el invertir una mayor cantidad de
recursos en la protección y evaluación de la seguridad de estos productos. Por desgracia, la seguridad de 
estos es en numerosas ocasiones dejada de lado debido a diferentes motivos como las 
grandes restricciones de rendimiento y de entorno normalmente asociadas a dichos dispositivos o los intentos por parte
de los fabricantes de abaratar costes de producción en productos low-cost del internet de las cosas. \\

Esta misma tendencia se puede observar respecto a la técnica del fuzzing, donde inputs especiales son 
fabricados y proporcionados a programas con el fin de encontrar fallos en su funcionamiento.
Aunque el fuzzing se ha consolidado en los últimos años como una técnica estándar en la industria del 
software gracias a su gran capacidad para encontrar fallos y vulnerabilidades, el fuzzing orientado a 
dispositivos IoT y por ende, dispositivos empotrados es un campo de investigación considerablemente 
reciente en el que aún se están dando los primeros pasos. Sumando a las causas descritas anteriormente, 
esto se debe a que aplicar técnicas de fuzzing a este tipo de dispositivos supone nuevos retos como la 
dificultad de obtener feedback sobre la ejecución de los bloques básicos de código sin disponer del código
fuente original.

En este proyecto se investigarán y aplicarán diversos enfoques y técnicas estado del arte con el fin de 
conocer el estado actual de esta novedosa rama del fuzzing orientado a dispositivos IoT.
\cleardoublepage

\begin{center}
	{\large\bfseries Introduction to Fuzzing, use cases and strategies 
	to find vulnerabilities in IoT devices}\\
\end{center}
\begin{center}
	Sergio García Cabrera\\
\end{center}
\vspace{0.5cm}
\noindent{\textbf{Keywords}: \textit{Fuzzing, IoT, Emulation, Vulnerability, Embedded systems,
,Black-box fuzzing, AFL++, QEMU, Open Source.}
\vspace{0.7cm}

\noindent{\textbf{Abstract}\\
Given the ever-increasing rise in the number of devices connected to the internet around us and the fact 
that nowadays these are made to handle more sensible information, it is clear that it has become a 
necessity to invest more time and resources in the protection and evaluation of the security of these 
products. Unfortunately, the security measures found on these are often left aside due to reasons such as 
the performance and environment constraints commonly associated with these kinds of devices or manufacturers
trying to reduce production costs of low-cost IoT devices.\\

The same trend can be observed regarding fuzzing, a technique where special crafted inputs are given to 
programs with the intention of finding bugs. Even though fuzzing has consolidated as an industry 
standard thanks to its great success finding bugs and vulnerabilities in code, IoT oriented 
fuzzing and, by extension, embedded oriented fuzzing is a research field that is not nearly as mature.
Adding up to the aforementioned issues, applying fuzzing techniques to this kind of devices comes with new 
challenges such as the difficulty to get feedback about execution of basic code blocks due to the lack of 
original source code.

In this thesis, different state-of-the-art techniques and approaches will be discussed and applied in order
to learn about the current state of such a novel research field that is IoT oriented fuzzing.
\cleardoublepage

\thispagestyle{empty}

\noindent\rule[-1ex]{\textwidth}{2pt}\\[4.5ex]

D. \textbf{Gustavo Romero López}, Profesor(a) del ...

\vspace{0.5cm}

\textbf{Informo:}

\vspace{0.5cm}

Que el presente trabajo, titulado \textit{\textbf{Introducción al Fuzzing, uso y estrategias de Fuzzing
para encontrar vulnerabilidades en dispositivos IoT}},
ha sido realizado bajo mi supervisión por \textbf{Sergio García Cabrera}, y autorizo la defensa de dicho trabajo ante el tribunal
que corresponda.

\vspace{0.5cm}

Y para que conste, expiden y firman el presente informe en Granada a Junio de 2022.

\vspace{1cm}

\textbf{El/la director(a)/es: }

\vspace{5cm}

\noindent \textbf{Gustavo Romero López}

\chapter*{Agradecimientos}
Quiero agradecer a mi familia por haberme acompañado y apoyado durante todo este viaje, especialmente
en esos momentos en los que creía que no sería capaz de seguir adelante. A mis amigos y mi pareja por 
ayudarme cuando tanto necesitaba desconectar y despejarme y por último a mi tutor, Gustavo Romero López
por orientarme pacientemente cuando ni siquiera sabía cómo enfocar el TFG.



