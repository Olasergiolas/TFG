\chapter{Conclusiones y trabajos futuros}
\label{conclusiones}
Una vez finalizado el proyecto, llega el momento de llevar a cabo una reflexión sobre el trabajo realizado y las metas alcanzadas. Al 
inicio del documento, se decía que el principal objetivo era investigar y aún más importante, aprender sobre un campo tan novedoso como 
el fuzzing IoT y una vez llegados a este punto podemos considerar que se ha cumplido satisfactoriamente nuestro objetivo.\bigskip

El conocimiento 
que se ha obtenido del tema a tratar además del aprendizaje de otras tecnologías colaterales como la ingeniería inversa, la 
emulación de código o la instrumentación dinámica resulta de gran valor como parte de la formación continua de alguien que desea seguir 
formándose en el campo de la ciberseguridad. Respecto al resto de objetivos, podemos concluir que también se ha llevado a cabo un buen
trabajo, habiendo tratado tanto en la teoría como en la práctica las técnicas más punteras respecto a la temática que nos atañe, 
además de haber contribuido a la comunidad mediante el reporte de bugs en proyectos como Qiling y AFL++ y haber creado un contenedor 
Docker que facilite al usuario a dar sus primeros pasos en el fuzzing de arquitectura cruzada. Y aunque es cierto que en el tiempo dedicado
a la realización del proyecto no hemos podido colaborar con desarrolladores de software ni fabricantes reportando vulnerabilidades reales 
previamente no descubiertas, el conocimiento adquirido resulta de gran ayuda para poder llevar a cabo esta tarea en el futuro.\bigskip

También es cierto que se han cometido errores, mayormente a la hora de planificar de forma razonable y desde un principio las tareas a
realizar dado el tiempo del que se disponía. Era habitual durante el desarrollo del proyecto encontrar una nueva herramienta o tecnología que podría 
resultar de gran interés y querer hacer uso de esta, llegando a un punto en el que tuvo que reducirse el contenido a tratar para el trabajo
por falta de tiempo.\bigskip

Acerca de la base de conocimiento con la que se inició el proyecto, dada una temática más cercana al bajo nivel en la que hemos tratado conceptos como la gestión de memoria, el desensamblado
de binarios o la emulación de código, es de esperar que asignaturas como ''Estructura de Computadores'' o ''Sistemas Operativos'' hayan proporcionado una buena base 
que ha facilitado el asimilar rápidamente nuevos conceptos, además de servir de gran ayuda para identificar muchos de los problemas 
encontrados durante el proyecto. También es necesario destacar la utilidad de las asignaturas de ''Sistemas Empotrados'' e
''Infraestructura Virtual'', habiendo la primera aportado una base sobre aspectos como trabajar con C orientado a sistemas empotrados o 
el uso de librerías estándar de C alternativas mientras que la segunda ha sido clave para la creación del contenedor Docker, el uso de 
integración continua para este y el aprendizaje de buenas prácticas de trabajo en Github.\bigskip

También podemos concluir tras haber llevado a cabo los experimentos realizados que el fuzzing orientado a IoT y sistemas empotrados no 
está preparado todavía para alcanzar el público general debido a la complejidad de los retos asociados con esta técnica, al contrario de 
lo que sucede con el fuzzing general de código que cada vez está más estandarizado.\bigskip

Por último, como hemos comentado anteriormente ha habido una serie de contenidos y tecnologías de gran interés que no han podido ser 
incluidas en este trabajo por falta de tiempo. De cara al futuro se planea expandir los conocimientos adquiridos investigando el 
fuzzing IoT fuera de los dispositivos basados en Linux, por ejemplo sobre firmware baremetal o basado en otros sistemas operativos 
como FreeRTOS. Además, otro experimento que sin lugar a duda será realizado en el futuro es la extracción del firmware de un dispositivo
empotrado mediante el volcado del contenido de su memoria flash.