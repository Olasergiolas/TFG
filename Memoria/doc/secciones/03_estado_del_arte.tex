\chapter{Estado del arte}
\label{estado_del_arte}
Tradicionalmente, en el fuzzing orientado a dispositivos IoT no se aplicaban técnicas
específicas para este tipo de dispositivos, es decir, el principal foco de atención era el fuzzing de sus portales web a través 
de peticiones HTTP(s) con un enfoque de caja negra. Este tipo de fuzzers conocidos como ''dumb fuzzers'' tratan al software como si de una caja negra se tratase, 
algo que recibe un input y genera un output. El código que se haya alcanzado para procesar el input internamente no es tenido en
cuenta a la hora de modificar los inputs. El fuzzing de caja negra sigue un flujo de ejecución simple el cual comienza con la mutación de un input válido.
A partir de ahí este input mutado es utilizado como parámetro del objetivo a fuzzear, si el 
objetivo (web, binario, etc.) produce un timeout o crashea, el fuzzer registra el input que 
ha provocado este comportamiento y si el funcionamiento del objetivo es correcto se vuelve a empezar. Dentro del ''dumb fuzzing'' hay también
distintos niveles de complejidad, algunos autores proponen ''random fuzzers'' extremadamente básicos que 
llevan a cabo sus modificaciones al input de forma completamente aleatoria. Este enfoque utilizado por 
herramientas como\cite{zzuf}, sigue la filosofía que dio lugar originalmente al fuzzing en 
los años 80 de la mano de Barton Miller\cite{Miller1990}. Modificar inputs de forma completamente aleatoria
ejemplifica el principal problema del ''dumb fuzzing'', la baja cobertura de código. Para conseguir encontrar 
el mayor número de fallos en el código, es necesario intentar ejecutar el mayor porcentaje de este posible. Cuando 
un input está siendo generado de manera completamente aleatoria, es bastante probable que este no cumpla con
ciertas comprobaciones que el software realice sobre el formato del input. Esto provoca que el input sea descartado 
de manera prematura en la ejecución del software y no se lleguen a alcanzar secciones críticas de código.
Con el objetivo de intentar hacer frente a este reto, otros autores proponen lo que es 
conocido como ''Mutation-based fuzzing'' donde aún aplicando todavía un enfoque de caja negra, se infiere
información sobre el input original a partir de la identificación de patrones y se aplican heurísticas para generar
nuevas mutaciones. Estas heurísticas pueden ser modificables en tiempo de ejecución como hace Radamsa\cite{radamsa}
para conseguir una mayor tasa de éxito de cara a encontrar vulnerabilidades. Esta herramienta ejemplifica que incluso 
un buen fuzzer de caja negra es capaz de identificar de forma efectiva vulnerabilidades críticas en productos comerciales 
ampliamente utilizados como Cisco AnyConnect, Mozilla Firefox o Google Chrome.

Otros fuzzers de caja negra categorizados como ''Generation-based fuzzers''\cite{Felderer2016} van un paso más allá y utilizan diccionarios o modelos para solo 
mutar determinados campos de forma que se maximice el número de inputs ''válidos'' generados para la aplicación a fuzzear. Esto es de 
especial interés en aplicaciones que utilizan datos con una sintaxis especialmente compleja o verbose como HTML, SQL o XML.
Boofuzz\cite{boofuzz} es un fuzzer que implementa esta idea permitiendo al usuario crear scripts de Python en los que 
se defina un formato a seguir para el input y los campos de información que pueden ser fuzzeados sin dar lugar a inputs 
inválidos. Usando esta herramienta, el equipo de ''Security for Everyone'' descubrió un 0-day (CVE-2020-29596)\cite{securityforeveryone}
en MiniWeb HTTP server, un servidor HTTP básico orientado a dispositivos empotrados por su bajo uso de recursos.
Xiaotao et al.\cite{snipuzz} proponen ''Snipuzz'', una técnica de fuzzing de caja negra orientada a IoT en la que el fuzzer es capaz de obtener feedback 
de las respuestas que devuelve el dispositivo. La idea principal es poder deducir qué código ha sido ejecutado internamente en base a 
la respuesta obtenida al realizar una petición, aunque esto suponga depender de que el fabricante utilice mensajes de respuesta descriptivos.
Gracias a esta información adicional, ''Snipuzz'' es capaz de identificar qué rol cumple cada 
byte de un input y cómo afecta su modificación a la respuesta del dispositivo. Otros autores hacen frente al reto de conseguir generar inputs 
válidos a través de la invocación de métodos internos de las propias aplicaciones móviles de los fabricantes de dispositivos IoT para generar los 
inputs que enviar al dispositivo.
Basados en este planteamiento surgen ''IoTFuzzer''\cite{Chen2018} y ''DIANE''\cite{Redini2021}, dos fuzzers IoT que delegan la creación de mensajes 
a las aplicaciones para smartphone de fabricantes como TP-Link o Belkin diseñadas para gestionar remotamente los dispositivos IoT. 
Se trata de un enfoque interesante a tener en cuenta ya que estas aplicaciones siempre van a generar mensajes que respeten el formato esperado
por el dispositivo receptor. Partiendo de dicho concepto, estos fuzzers analizan automáticamente el código de las apps móviles en busca de secuencias de 
métodos que envíen mensajes al dispositivo y mediante instrumentación dinámica, ejecutan estos métodos cambiando el valor de sus parámetros. La principal diferencia 
entre ambos reside en el componente de la app móvil que toman como punto de partida. ''IoTFuzzer'' parte de los métodos a nivel de interfaz gráfica de la app
para introducir la información mutada de forma similar a como lo haría un usuario real. Dicho planteamiento presenta el inconveniente de que este tipo 
de apps suelen sanitizar el input del usuario, por lo que una gran cantidad de las mutaciones serán descartadas sin siquiera salir de la app móvil.
''DIANE'' soluciona el problema tomando como punto de partida aquellos métodos que sean ejecutados después de la sanitización de los inputs pero antes de
que se realice el envío del mensaje al dispositivo IoT. Por desgracia, las técnicas de fuzzing de caja negra que trabajan sobre el dispositivo IoT
directamente, ya sea haciéndole peticiones o ejecutando código en el hardware, suponen un gran sacrificio respecto a rendimiento ya que un hardware 
tan limitado como el encontrado en dispositivos empotrados nunca será capaz de aportar una alta tasa de ejecuciones/respuestas por segundo, siendo esto un 
factor clave a la hora de reducir el tiempo puede tardar un fuzzer en detectar una vulnerabilidad.

En la búsqueda por solucionar algunos de los problemas del fuzzing de caja negra como su baja eficiencia o su limitada cobertura de código, se 
adoptan enfoques de caja blanca y caja gris. El primero gira entorno a la idea de generar binarios instrumentados a partir del código fuente original. De esta forma, un compilador 
especial inserta código adicional encargado de reportar al fuzzer qué bloques básicos de código han sido ejecutados exactamente para qué inputs.
Aunque esta es la metodología más popular para realizar fuzzing de binarios en la actualidad utilizando herramientas como AFL++\cite{afl++} o 
Honggfuzz\cite{honggfuzz}, su aplicación orientada al internet de las cosas no es viable en la mayoría de casos ya que como se comentó en 
''\nameref{introduccion}'' los componentes software utilizados en firmware IoT no suelen ser de código abierto, por lo que no pueden ser recompilados 
usando compiladores que instrumenten el código. Por suerte, AFL++ también implementa distintos modos de funcionamiento de caja gris
(QEMU\cite{qemuafl}, FRIDA\cite{frida}, Unicorn\cite{unicorn}\dots) que serán tratados a continuación.\bigskip

Es la dificultad para conseguir acceso al código fuente original lo que hace que un enfoque como el fuzzing de caja gris resulte mucho más atractivo 
cuando se busca conseguir una mayor cobertura de código al fuzzear binarios de los cuales no se posee el código fuente. En este caso, los fuzzers obtienen 
feedback sobre el estado interno de la ejecución del software sin necesidad de instrumentar el código fuente original. Conseguir información que pueda ser
indicativa del estado interno de la ejecución de un software es el principal reto de los fuzzers de caja gris. 
Llevar a cabo este análisis dinámico de software en arquitecturas más comúnmente orientadas a 
propósito general como es x86-64 no es tarea difícil, pero hacerlo sobre plataformas altamente limitadas tanto en recursos como en funcionalidad
puede suponer un reto. De esta forma surge la idea de añadir una capa de abstracción a través de emulación con soluciones basadas en QEMU\cite{qemu}
que permitan realizar instrumentación dinámica de binarios. QEMU es un eficiente emulador de código abierto capaz de correr sistemas operativos y binarios
diseñados para arquitecturas como ARM en otras completamente distintas como x86-64. Aunque en muchas ocasiones el concepto de emular software se asocia a
una gran reducción de rendimiento, es necesario tener en cuenta que un ordenador de propósito general moderno como un portátil o un sobremesa es 
considerablemente más potente que la mayoría de sistemas empotrados actuales, por lo que aún habiendo un mayor overhead en la ejecución el impacto de este es mitigado.
Los resultados de las investigaciones llevadas a cabo por Muench et al.\cite{Muench2018} muestran cómo emulando un sistema empotrado en su totalidad 
se consigue una mejora de rendimiento sobre el hardware original.
Cuando hablamos de técnicas de emulación de sistemas empotrados, podemos aplicar la siguiente clasificación:
\begin{enumerate}[I]
    \item \textbf{User-mode emulation}: Se emula exclusivamente la ejecución del binario que resulte de interés. QEMU evita tener que emular el sistema 
    operativo al completo traduciendo las llamadas al sistema de la aplicación emulada en llamadas al sistema del host. Es por ello que usar este modo 
    solo es factible si tanto el host como el huésped comparten sistema operativo. Herramientas como Qemuafl\cite{qemuafl}, un fork de QEMU modificado para añadir 
    integración con AFL++, consiguen un mayor número de 
    ejecuciones por segundo en comparación con (II), pero por desgracia la ausencia de emulación de ciertos dispositivos hardware puede hacer imposible 
    el correcto funcionamiento del binario que está siendo emulado. Zheng et al.\cite{Zheng2019} destacan que durante sus intentos de emular distintos 
    servidores HTTP, DNS y SSH utilizados en routers comerciales, este modo de QEMU fue incapaz de emular correctamente ninguno de los binarios analizados.
    \item \textbf{System-mode o Full system emulation}: Se trata de una técnica implementada por QEMU en la cual se crea una máquina virtual que emula un 
    sistema al completo, esto incluye CPU, sistema operativo, periféricos hardware, etc. Gracias a esta técnica es posible emular software que puede resultar 
    problemático aplicando (I) debido a posibles dependencias duras sobre otros dispositivos. Un gran número de autores proponen el uso de esta técnica 
    para la ejecución de software IoT fuera de su hardware original debido a que permite alcanzar un balance entre rendimiento y estabilidad. La idea es que 
    una vez el sistema al completo es emulado, es posible fuzzear un binario del firmware aplicando el enfoque que se desee, ya sea de caja negra, gris o 
    blanca (si se dispone del código fuente). Ejemplo de uso de esta técnica de emulación es FIRMADYNE\cite{Chen2016}, un proyecto basado en QEMU\cite{qemu}
    que facilita la ejecución e instrumentación de firmware IoT a través de emulación system-mode con un kernel modificado (soporte para ARM y MIPS), la 
    implementación de un extractor de firmware y de una librería para simular una NVRAM real. Aunque se trata de un concepto interesante, diversos autores 
    han demostrado como FIRMADYNE\cite{Chen2016} fracasa a la hora de emular correctamente la mayoría de firmware IoT basado en Linux, con tasas de éxito del
    $\sim$16\%\cite{Kim2020} de 1124 imágenes firmware puestas a prueba pertenecientes a distintos routers y cámaras IP. Mingeun et al.\cite{Kim2020} 
    sugiere que en la mayoría de los casos, los fracasos de FIRMADYNE\cite{Chen2016} vienen dados por pequeños fallos de configuración fácilmente 
    solucionables. Es por ello que proponen FirmAE, un emulador IoT que aplica heurísticas capaces de detectar fallos de configuración propios de cada
    firmware y solucionarlos. Gracias a esto se consigue una tasa de éxito del $\sim$80\% con respecto a las mismas imágenes firmware.
    Zhang et al.\cite{Zhang2021} hacen uso de ambas herramientas durante su investigación para emular firmware IoT como paso previo a la aplicación de
    fuzzing a las interfaces web de los dispositivos.
    \item \textbf{Unicorn Engine}: Unicorn\cite{unicorn} es un framework basado en QEMU que propone un enfoque de emulación ultraligero en el que el elemento único a 
    emular es la CPU. Se diferencia de (I) en que Unicorn no realiza traducción de llamadas al sistema ni gestión de señales POSIX, solo traduce instrucciones 
    máquina de la arquitectura del huésped a instrucciones comprensibles por la CPU del host. Además, proporciona una API intuitiva que facilita 
    considerablemente operaciones necesarias para la instrumentación dinámica de binarios como lecturas/escrituras de memoria y registros, mapeos de memoria o 
    la posibilidad de definir hooks que serán ejecutados al alcanzarse ciertas direcciones de memoria. Como ya se ha comentado, AFL++ dispone de integración 
    con este framework el cual posibilita la aplicación de fuzzing sobre funcionalidades específicas de binarios complejos. Véase un binario que obtenga su 
    input directamente desde un dispositivo hardware como una antena de radio o un chip NFC, en lugar de fuzzear el binario al completo incluyendo todo el 
    proceso de tratamiento de señales se instrumenta dinámicamente el binario para definir un nuevo punto de entrada del código y ajustar los registros
    adecuadamente para poder así fuzzear únicamente la sección de código que nos interese. Qiling\cite{qiling} es otro framework de emulación basado en
    Unicorn\cite{unicorn} que intenta combinar las ventajas de (III) con las de (I). Esto significa poder emular binarios disponiendo de soporte 
    para llamadas al sistema, librerías dinámicas, I/O y otros conceptos de alto nivel pudiendo aplicar instrumentación dinámica a través de una API fácil 
    de usar, además de poder dejar atrás limitaciones como la necesidad de que host y huésped compartan SO.
    \item \textbf{Augmented process emulation}: Técnica propuesta por Zheng et al.\cite{Zheng2019} que implementan en su fuzzer IoT Firm-AFL. El objetivo 
    principal es combinar la emulación de binarios mediante (I) durante la mayor parte del tiempo y poder de forma dinámica cambiar a (II) si se es 
    requerido durante la ejecución.
    \item \textbf{Hardware in-the-loop}: Utilizan tanto emulación como el hardware real del dispositivo para la ejecución de código. Para implementar esta 
    técnica se suele recurrir a algún tipo de proxy capaz de redirigir al dispositivo ciertas operaciones que tengan una fuerte dependencia en el hardware
    original mientras que la mayoría de instrucciones máquina se ejecutan en la CPU de otro sistema. Aplicar un enfoque hardware in-the-loop aporta a costa 
    de reducir escalabilidad las ventajas de usar emulación manteniendo a su vez la alta estabilidad que proporciona usar el hardware original.
    AVATAR\cite{Zaddach2014} aplica esta técnica delegando las operaciones de I/O al hardware original mientras que el resto de la ejecución se lleva a 
    cabo mediante emulación. uAFL\cite{uAFL} lleva a cabo la ejecución del software en el hardware original y extrae del dispositivo a través de JTAG
    información de cobertura de código gracias al hardware de debugging llamado ''ARM ETM'' incluido en ciertos procesadores ARM. Dado que 
    esta información necesita de una gran cantidad de recursos para ser procesada, se extrae del dispositivo para su procesamiento y se utiliza el resultado
    para guiar el proceso de fuzzing.
\end{enumerate}

[TBD si hay tiempo: dificultad de que puede que algo que crashee en el hardware original no lo haga al emular...]

Más allá de la emulación, también existen otras tecnologías comúnmente utilizadas en conjunción con el fuzzing IoT como son los ''Sanitizers'' y los
''Rewriters''. En primer lugar, los sanitizers como ASAN\cite{Serebryany2012} o MSAN\cite{Stepanov2015} son herramientas que buscan ayudar con la detección de errores de memoria mediante la instrumentación 
de código. Al instrumentar el código de una aplicación, son capaces de añadir mecanismos de protección al stack y al heap además de reemplazar funciones
de acceso a memoria como \textit{malloc} o \textit{free} con versiones propias capaces de reportar información detallada sobre la operación que está siendo 
realizada. El objetivo principal de todo esto es poder monitorizar operaciones ilegales de memoria como buffer overflows,
use-after-free, leaks de memoria, etc. introduciendo el mínimo overhead posible, además de poder hacer que el binario crashee en cuanto se detecte una de estas
operaciones ilegales, incluso si durante un uso normal del software esto no hubiera sido suficiente como para afectar de forma visible su funcionamiento. 
Como es lógico, facilitar a los fuzzers la detección de errores de memoria resulta toda una ventaja pero debido a que la mayoría de sanitizers basan su
funcionamiento en la instrumentación de código fuente, su aplicación no resulta viable para el tipo de fuzzing que nos atañe. Con el objetivo de poder 
llevar esta tecnología a binarios de código cerrado surge QASAN\cite{Fioraldi2020}, un sanitizer que consigue imitar la funcionalidad de ASAN\cite{Serebryany2012} 
mediante las capacidades de instrumentación dinámica que proporciona QEMU. Como podremos observar en \nameref{implementacion}, aplicar esta herramienta 
junto a un fuzzer puede revelar bugs de memoria que no hubieran sido descubiertos aplicando fuzzing clásico. \bigskip

Respecto a los rewriters, se tratan de herramientas que buscan instrumentar estáticamente binarios, inyectando código de instrumentación directamente en 
el binario. Aunque existen multitud de rewriters siendo desarrollados actualmente como Retrowrite\cite{Dinesh2020}, e9Patch\cite{Duck2020} o 
ZAFL\cite{Nagy2021}, es necesario tener en cuenta que se tratan de herramientas muy recientes que en muchos casos pueden introducir fallos en los binarios
instrumentados además de carecer en ocasiones de funcionalidades básicas como soporte para binarios de diferentes arquitecturas más allá de x86-64. 
Dichos problemas hacen que la tecnología no esté todavía preparada para su uso a gran escala y menos aún en el contexto de los sistemas empotrados y el 
internet de las cosas.
\bigskip

Una vez ya tratados los distintos enfoques, técnicas y herramientas presentes en el panorama actual del fuzzing IoT se nos abre un gran abanico de 
posibilidades a la hora de poner en práctica y combinar el uso de algunas de estas técnicas y herramientas para intentar conseguir detectar el mayor 
número de vulnerabilidades en software orientado a sistemas empotrados e IoT. Esto será llevado a más adelante en el proyecto, donde pondremos a prueba 
conceptos como el fuzzing de caja negra, emulación system-mode y user-mode, instrumentación dinámica o los sanitizers. Por último, antes de finalizar con
esta sección cabe destacar que, aunque el fuzzing de protocolos como Zigbee o MQTT es también una rama de investigación de gran interés en el campo del
fuzzing IoT, su discusión queda fuera del alcance de este proyecto.

\bigskip
[TBD añadir tablas y diagramas]
