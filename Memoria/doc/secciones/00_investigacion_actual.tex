%%%%%%%%%%%%%%%%%%%%%%%%%%%%%%%%%%%%%%%%%
% Short Sectioned Assignment LaTeX Template Version 1.0 (5/5/12)
% This template has been downloaded from: http://www.LaTeXTemplates.com
% Original author:  Frits Wenneker (http://www.howtotex.com)
% License: CC BY-NC-SA 3.0 (http://creativecommons.org/licenses/by-nc-sa/3.0/)
%%%%%%%%%%%%%%%%%%%%%%%%%%%%%%%%%%%%%%%%%

% \documentclass[paper=a4, fontsize=11pt]{scrartcl} % A4 paper and 11pt font size
\documentclass[11pt, a4paper, openany]{report}
\usepackage[T1]{fontenc} % Use 8-bit encoding that has 256 glyphs
\usepackage[utf8]{inputenc}
\usepackage{fourier} % Use the Adobe Utopia font for the document - comment this line to return to the LaTeX default
\usepackage{listings} % para insertar código con formato similar al editor
\usepackage[spanish, es-tabla]{babel} % Selecciona el español para palabras introducidas automáticamente, p.ej. "septiembre" en la fecha y especifica que se use la palabra Tabla en vez de Cuadro
\usepackage{url} % ,href} %para incluir URLs e hipervínculos dentro del texto (aunque hay que instalar href)
\usepackage{graphics,graphicx, float} %para incluir imágenes y colocarlas
\usepackage[gen]{eurosym} %para incluir el símbolo del euro
\usepackage{cite} %para incluir citas del archivo <nombre>.bib
\usepackage{enumerate}
\usepackage{hyperref}
\usepackage{graphicx}
\usepackage{tabularx}
\usepackage{booktabs}

\usepackage[table,xcdraw]{xcolor}
\hypersetup{
	colorlinks=true,	% false: boxed links; true: colored links
	linkcolor=black,	% color of internal links
	urlcolor=cyan		% color of external links
}
\renewcommand{\familydefault}{\sfdefault}
\usepackage{fancyhdr} % Custom headers and footers
\pagestyle{fancyplain} % Makes all pages in the document conform to the custom headers and footers
\fancyhead[L]{} % Empty left header
\fancyhead[C]{} % Empty center header
\fancyhead[R]{Sergio García Cabrera} % My name
\fancyfoot[L]{} % Empty left footer
\fancyfoot[C]{} % Empty center footer
\fancyfoot[R]{\thepage} % Page numbering for right footer
%\renewcommand{\headrulewidth}{0pt} % Remove header underlines
\renewcommand{\footrulewidth}{0pt} % Remove footer underlines
\setlength{\headheight}{13.6pt} % Customize the height of the header

\usepackage{titlesec, blindtext, color}
\definecolor{gray75}{gray}{0.75}
\newcommand{\hsp}{\hspace{20pt}}
\titleformat{\chapter}[hang]{\Huge\bfseries}{\thechapter\hsp\textcolor{gray75}{|}\hsp}{0pt}{\Huge\bfseries}
\setcounter{secnumdepth}{4}
\usepackage[Lenny]{fncychap}


\begin{document}

	\chapter{Estado del arte y planificación}

    En este breve documento de carácter temporal se resumirán ideas y contenidos investigados 
    hasta el momento sobre el \textbf{Estado del arte} del \textbf{Fuzzing en dispositivos IoT}.

    \section{Introducción}
    Siendo el fuzzing una técnica de testeo de software ya consolidada con un gran listado de herramientas e información
    al respecto, cuando hablamos de fuzzing orientado a dispositivos IoT encontramos que actualmente se trata de
    un campo de investigación muy reciente para el que no se dispone apenas de herramientas especializadas o de
    información más allá de papers publicados en el último año (en muchas ocasiones teóricos o con poca utilidad real).

    \section{Puntos clave}
    \begin{itemize}
        \item Actualmente, la gran mayoría de software dedicado específicamente a fuzzing IoT 
        provienen de papers y son proyectos poco documentados y/o abandonados
        (véase \href{https://github.com/XtEsco/Snipuzz}{Snipuzz} o \href{https://github.com/zyw-200/FirmAFL}{FirmAFL}). 
        \item Las principales técnicas que pueden ser aplicadas a fuzzing IoT son:
        \begin{enumerate}
            \item \href{https://rehosti.ng/}{Rehosting}
            \item Fuzzing de protocolo de comunicación (MQTT y similares).
            \item Fuzzing de peticiones HTTP a portal web de administración.
            \item Fuzzing a snippets de código con Unicorn Engine.
            \item Binary-only fuzzing en QEMU.
        \end{enumerate}
        \item La mayoría de firmware IoT no es Open Source y recientemente es distribuido como binarios 
        cifrados.
        \item Mayoritariamente aplicaremos Black-box o Gray-box fuzzing.
        \item En muchas ocasiones es complicado emular firmware IoT debido a las dependencias hardware.
    \end{itemize}

    \section{Herramientas}
    
    Herramientas encontradas hasta el momento que pueden ser de utilidad.
    \begin{itemize}
        \item \href{https://github.com/pr0v3rbs/FirmAE}{FirmAE}: Emulador de firmware para dispositivos
        como routers o cámaras wifi basado en QEMU. Inspirado en Firmadyne pero con mejor compatibilidad, soporte
        para levantar el portal web del dispositivo y opciones de depuración (gdb y shell). La idea es
        utilizarlo para fuzzear los portales web.
        \item \href{https://www.qemu.org/}{QEMU}: Emulador multiplataforma. La idea es utilizarlo junto a AFL++ para 
        realizar el black-box fuzzing de binarios extraídos de firmware con binwalk.
        \item \href{https://github.com/jtpereyda/boofuzz}{Boofuzz}: Framework de fuzzing. Con esto se podría
        desarrollar un fuzzer HTTP para los portales web de los dispositivos.
        \item \href{https://github.com/AFLplusplus/AFLplusplus}{AFL++}: Fuzzer que podría usarse en 
        modo QEMU para el binary-only fuzzing.
        \item \href{https://github.com/eclipse/iottestware.fuzzing}{iottestware.fuzzing}: Fuzzing de MQTT y CoAP.
        Aunque parece ser un proyecto abandonado la idea es interesante.
        \item \href{https://github.com/ReFirmLabs/binwalk}{Binwalk}: Para desenpaquetar firmware.
        \item \href{https://github.com/unicorn-engine/unicorn}{Unicorn Engine}: Proyecto hermano de QEMU más centrado en
        emular la ejecución de instrucciones de CPU que en emular el sistema al completo. Esto puede ser interesante si
        queremos fuzzear un binario que tenga dependencias (hardware, entorno, periféricos, etc.) que no podamos
        satisfacer, en su lugar, extraemos del binario el fragmento de código que nos interese y fuzzeamos solo eso.
        \item \href{https://gitlab.com/akihe/radamsa}{Radamsa}: Generador de test cases. Partiendo de un input que
        se le proporcione, genera variaciones de este pero sin llegar a probarlas sobre el fuzzing target. Puede ser
        útil para obtener una serie de test cases que utilizar como input para algún script que se desarrolle. 
    \end{itemize}

    \section{Estructuración}
    En esta sección se comentará de forma general qué se plantea tratar en el TFG actualmente.
    
    \begin{itemize}
        \item Introducción al fuzzing en general
        \item Estado del arte
        \item Planificación
        \item Fuzzing targets a utilizar
        \item Posibles casos prácticos de aplicación de técnicas de fuzzing IoT:
        \begin{itemize}
            \item \textbf{Rehosting + Fuzzing HTTP del portal web}: Usar emulación de firmware
            para levantar el portal web del dispositivo y realizarle fuzzing con las peticiones HTTP.
            \item \textbf{Fuzzing de protocolos de comunicación IoT como MQTT o CoAP}
            \item \textbf{Fuzzing de binarios del firmware (AFL QEMU)}: Uso de binwalk para extraer
            binarios propios del firmware a los que aplicar black-box fuzzing con AFL.
            \item \textbf{Ingeniería inversa + fuzzing de snippets de código (Unicorn Engine)}:
            Pasar alguno de los binarios del firmware por Ghidra, extraer algún fragmento de código
            relevante y realizarle fuzzing.
        \end{itemize}
        \item Desarrollo de algún contenedor o herramienta para facilitar/automatizar el proceso
        \item Resultados y comparaciones
        \item Conclusiones
    \end{itemize}

    \section{Recursos útiles}
    \begin{itemize}
        \item \href{https://github.com/andreia-oca/awesome-embedded-fuzzing}{Recopilatorio de información sobre
         IoT fuzzing}.
        \item \href{https://blog.attify.com/fuzzing-iot-devices-part-1/}{Caso práctico de uso de AFL++ sobre un 
        binario extraído de firmware IoT}.
         \item \href{https://www.youtube.com/playlist?list=PLhixgUqwRTjy0gMuT4C3bmjeZjuNQyqdx}{Uso de fuzzing sobre
        "sudo" para llegar a CVE-2021-3156}.
        \item \href{https://fuzzing-project.org/tutorials.html}{The Fuzzing Project}.
        \item \href{https://github.com/google/fuzzing/blob/master/docs/glossary.md}{Glosario de términos de fuzzing}.

    \end{itemize}
    
\end{document}
